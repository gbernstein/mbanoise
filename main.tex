%\documentclass[onecolumn]{aastex631}
\documentclass[linenumbers, onecolumn]{aastex631}

%\defcitealias{H21}{H21}

\newcommand{\ra}{\mathrm{RA}}
\newcommand{\dec}{\mathrm{Dec}}
\newcommand{\lsst}{\textit{LSST}}
\newcommand{\gaia}{\textit{Gaia}}
\newcommand{\AU}{\mathrm{au}}
\newcommand{\eqq}[1]{Equation~(\ref{#1})}
\newcommand{\ie}{\textit{i.e.\/}}
\newcommand{\eg}{\textit{e.g.\/}}
\newcommand\edited[1]{{\color{blue} {#1}}}
\newcommand\gary[1]{{\color{red} {\textbf{GMB}: #1}}}
% Turn off change highlighting
%\newcommand\edited[1]{#1}

\newcommand{\vecI}{\mathbf{I}}
\newcommand{\vecb}{\mathbf{b}}
\newcommand{\vece}{\mathbf{e}}
\newcommand{\bhat}{\mathbf{\hat b}}
\newcommand{\rhat}{\mathbf{\hat r}}
\newcommand{\phat}{\boldsymbol{\hat\phi}}
\newcommand{\uhat}{\boldsymbol{\hat u}}
\newcommand{\zhat}{\mathbf{\hat z}}
\newcommand{\vecp}{\mathbf{p}}
\newcommand{\vecq}{\mathbf{q}}
\newcommand{\vecr}{\mathbf{r}}
\newcommand{\vecv}{\mathbf{v}}
\newcommand{\vecx}{\mathbf{x}}


\newcommand{\matA}{A}
\newcommand{\matB}{B}
\newcommand{\matC}{C}

\newcommand{\nIr}{\left\langle nI_r^2\right\rangle}
\newcommand{\nIphi}{\left\langle nI_\phi^2\right\rangle}
\newcommand{\nIz}{\left\langle nI_z^2\right\rangle}

\newcommand{\Var}{\textrm{Var}}
\newcommand{\vcirc}{v_c}
\newcommand{\vrel}{v_{\rm r}}
\newcommand{\Msun}{M_\odot}
\newcommand{\covm}{C}
\newcommand{\lop}{\varpi}
\newcommand{\Lhat}{\hat L}
\usepackage{natbib} 
\usepackage{amsmath}
\usepackage{enumitem}
\usepackage{verbatim}
\usepackage{graphicx}
\usepackage{subfigure}
\usepackage{color}
\usepackage{xcolor}
\usepackage{float}
\usepackage{hyperref}
%\usepackage{lineno}
%\linenumbers

\shorttitle{Asteroid Brownian motion}

\begin{document}

\title{Brownian motion of main-belt asteroids on human timescales} 

\author[0000-0002-8613-8259]{Gary M. Bernstein}
\affiliation{Department of Physics and Astronomy, University of Pennsylvania, Philadelphia, PA 19104, USA}
\email{garyb@physics.upenn.edu}
\correspondingauthor{Gary M. Bernstein}


\begin{abstract}
  xxx
\end{abstract}

\keywords{asteroids}

\section{Introduction}

High-precision measurements of positions of Solar System bodies have
great value to multiple scientific pursuits.  They can reveal
gravitational accelerations due to undiscovered bodies such as a
hypothetical ``Planet X'' \citep[\eg][]{holmanP9,inpopP9,trojans,occultations} or passing primordial
black holes \citep{pbh}, determine
the masses of known bodies \citep[\eg][]{goffin,baer}, test the laws of physics, and characterize 
the non-gravitational forces that drive many forms of planetary
migration, \eg\ the maintainence of the near-Earth asteroid (NEA)
population \citep{yorp}.  While most applications of precision orbit determination
have used the 8 major planets as test particles, there are now $>10^6$
known minor planets available for this task.  These small bodies' astrometric
information content is growing extremely rapidly through large-scale
survey projects that can produce
milliarcsecond-scale accuracy for magnitude-limited small-body populations, such as Gaia \citep{gaiass3},
PanSTARRS,\footnote{\url{https://www2.ifa.hawaii.edu/research/Pan-STARRS.shtml}}
and particularly the upcoming Rubin observatory
\textit{Legacy Survey of Space and Time} (LSST).\footnote{\url{https://rubinobservatory.org/}}  The great majority
of currently-tracked objects are main-belt asteroids (MBAs), so we
wish to investigate the potential power of MBAs as gravitational test
bodies.

The most advanced ephemerides for the Solar System \citep{de440,inpop,pitjeva} consider the
gravitational forces emanating from the Sun, from the major planets and their large
moons, from up to $\approx300$ of the largest individual MBAs, and
from distributed mass annuli representing the smaller members of
 the asteroid belt(s) and Kuiper belt.
Here we ask: \textit{what is the RMS ephemeris error in angular or radial
position a typical MBA accrues because of its gravitational
encounters with other individual MBAs?}  We are interested not in the
perturbations that can be well described by a multipolar ring model
for the full population; rather the Brownian motion that accumulates
from closer encounters with: (a) MBAs that \emph{are not included} in the
ephemeris model at all; and (b) asteroids that \emph{are} included
in an ephemeris model, but have some uncertainty $\sigma_M$ in their masses.
Any such departures became a source of noise in any
inferences that we make from use of these MBAs as
dynamical tracer particles.  We aim to quantify this noise, as a
function of the RMS uncertainty $\sigma_M$ on the masses of individual
asteroids, and assuming that deflectors with $M<\sigma_M$ are not
included in the ephemeris at all.

In principal, one can collect all of the positional information for
all $>10^6$ known solar-system bodies, and fit them to a model with
a free state vector and mass for each body.  Calculating the influence
of $>7\times10^6$ free parameters on $10^8$ or so observables is,
however, unlikely to be computationally feasible, and most of the
bodies' masses will be lost in measurement noise and/or be degenerate
with other bodies' masses.  The most ambitious attempt at a global
solution that we are aware of is that of \citet{goffin}, who was able
to include 250 MBAs as gravitating bodies in the fit to the full
corpus of observations at the time.  For only $\approx 1/2$ of the
bodies did the fit yield masses $>3\times$ the uncertainties.  With
vast improvements in data and computing power since then and in the
next decade, the number of detectable
asteroid masses should increase, but there will always be
uncertainties that accrue from the unmodelled deflectors and the
finite $\sigma_M$ of modelled deflectors.  These remnant uncertainties
will in turn serve as noise in attempts to model new sources of
acceleration.

While long-term diffusion of orbital elements has been extensively
investigated in the context of planetary-system formation and
planetary migration, we address here the more limited question of how
much the MBAs will have their orbit elements and positions altered
by Brownian motion over a time period $T \lesssim100$~yr of human
observation, in the present dynamical environment.  We will make use
of the currently known population of MBAs, correcting for
incompleteness of current surveys at lower masses---or more precisely,
at fainter absolute magnitudes $H,$ since the $H$ distribution is
known far more accurately than the $M$ distribution.


\section{Calculation overview}
We aim to calculate the Brownian motion variance to an accuracy of a
factor $\approx 2.$
Greater accuracy in the calculation is not warranted
since some of the principal inputs have substantial uncertainty.  The
number of MBAs vs mass is 
uncertain because of unknowns in the conversion
from $H$ to mass; the status of future observations and $\sigma_M$ is
also unclear.  Hence we can determine our quantities only to leading
order in the MBA orbital eccentricity $e$ of the tracer population.
We will assume throughout that the joint
distribution of mass (or $H$) and orbital elements is separable into a
mass distribution and an orbital distribution.  In this scenario, the
Brownian motion is independent of the tracer mass, and involves an
integral over the mass distribution of the deflecting bodies.

We adopt a cylindrical coordinate system $(r,\phi,z),$ with $\zhat$
normal to the initial orbital plane, and $\phi=z=0$ toward the
perhelion of the initial orbit, \ie\ both the initial ascending node
$\Omega_0$ and longitude of perihelion $\lop_0$ equal to zero.  The
unit vectors $\rhat, \phat$ rotate with the target asteroid.  The mean
anomaly at time $t=0$ is $M_0.$

All distances will be given in units of the original semimajor axis $a_0$, and all velocities in units of the circular velocity $\vcirc \equiv \sqrt{G\Msun/a_0}.$  In these units, $G\Msun=1,$ the period of the initial orbit is $2\pi,$ and the (unperturbed) mean anomaly is $M=t+M_0.$ 

We describe all encounters with other asteroids in the impulse approximation, defining $\vecI$ as the $\Delta\vecv$ imparted on the target by the deflector.
In our units, the gravitational impulse imparted by a deflector of mass $M_d$ approaching at impact parameter $\vecb=b\bhat$ and relative velocity $\vrel$ is
\begin{equation}
  \vecI = 2 \frac{M_d}{\Msun} (bv)^{-1} \bhat.
  \label{eq:impulse}
\end{equation}
All of our results will be derived at leading order in $\vecI,$ which
is very well justified by the small size of MBA-induced impulses.

One of our tasks will be to derive, from the known asteroid population, the rate (per target) of encounters vs the imparted impulse,
\begin{equation}
  \frac{dN}{dt\,dI_r\,dI_\phi\,dI_z},
  \label{eq:dN}
\end{equation}
where the components of $\vecI$ are given in the cylindrical basis
vectors about the asteroid's position at the impulse.
We will estimate this function from the known population,
approximating it as constant within each of three subsets of tracer MBAs:
the inner, middle, and outer regions bounded by the 4:1, 3:1,
5:2, and 2:1 mean-motion resonances with Jupiter.
We will further assume that the impulses on a given target are drawn
independently from this distribution, \ie\ a Poisson process defined
by this rate.  In this case, the only properties of the impulse
distribution that we need are its second moments for components $\alpha \in \{r,\phi,z\}$
\begin{equation}
  \left \langle n I_\alpha^2 \right\rangle \equiv \int d^3I \frac{dN}{dt\,dI_r\,dI_\phi\,dI_z} I_\alpha^2.
\label{eq:nvsq}
\end{equation}
The average of the cross terms $I_rI_z$ and $I_\phi I_z$ vanishes if
the deflector distribution is symmetric in inclination as expected, and we find numerically that the mean $I_rI_\phi$ is small enough to ignore.

From this knowledge of the impulse distribution, our goal is to obtain
the covariance matrix of the deviations of the target's position from
the initial orbit, after some time $T.$ The position observables are
the range, plus the celestial latitude and longitude of the target.  We will simplify our results by assuming a heliocentric observer, so that the observational position vector is $\vecp\equiv (r,\theta=z/r,\phi).$ The quantity we seek is the covariance matrix $\covm^p$ of the observations attributable to the accumulated gravitational perturbations:
\begin{equation}
  \covm^p  \equiv  \left\langle  \Delta\vecp \Delta\vecp^T\right\rangle,  \label{eq:Cp}
\end{equation}
where the angle brackets indicate an average over possible
realizations of the impulse history, and over the mean anomaly $M$ at the time of observation.  To do so, we will introduce an intermediate set of 6 parameters $\vecq$ describing the deviations of the orbital elements induced by the impulses.  The $\vecq$ components will be selected to be equal to zero before any impulses are applied, and each responds linearly to $\vecI$ at $|I|\ll 1.$
We will derive the matrix $\matA$ that describes the orbital-element shifts at time $T$ that arise from an impulse at time $t_i$: 
\begin{equation}
  \Delta \vecq(T,t_i) = \matA(e_0, T, t_i) \cdot \vecI.
  \label{eq:A}
\end{equation}
In our first-order perturbation theory, $\vecq(T)$ will be the sum of the $\Delta\vecq_i$ imparted by all impulses applied at times $0<t_i<T.$  Because the impulses are uncorrelated, $\vecq$ will therefore be the result of a random walk.  The distribution will have a covariance matrix $\covm^q(T) \equiv \left\langle \vecq(T) \vecq^T(T) \right\rangle$ whose elements are
\begin{eqnarray}
  \covm^q_{jk}(e_o,T) & = & \left\langle \sum_{i,\gamma} \matA_{j\gamma}(e_0,T,t_i) \matA_{k\gamma}(e_0,T,t_i) I^2_{i,\gamma} \right\rangle \\
           & = & \sum_\gamma \int dt_i \matA_{j\gamma}(e_0,T,t_i) \matA_{k\gamma}(e_0,T,t_i) \left\langle n I_\gamma^2\right\rangle.
\label{eq:Cqjk}
\end{eqnarray}
In the first line, the sum $i$ runs over the impulses and $\gamma$ runs over the components $r,\phi,z$ of the impulse. The second line evaluates the expectation value of averaging over realizations of the random walk of impulses, exploiting the independence of the individual impulses from each other.

The last element of our calculation will be a conversion from the
element shifts $\vecq$ into the observed displacements $\Delta\vecp$
at time $T.$  In linear perturbation theory this will again be expressible as a matrix
\begin{equation}
  \Delta\vecp(e_0,T) = \matB(e_0,T) \cdot \vecq(T).
\label{eq:B}
\end{equation}
Combining this with \eqq{eq:Cp} and \eqq{eq:Cqjk} yields the desired result
\begin{eqnarray}
  \covm^p_{\alpha\beta}(e_0,T) & = & \sum_{jk} B_{\alpha j}(e_0,T) B_{\beta k}(e_0,T) \covm^q_{jk}(T) \nonumber \\
  & = & \sum_\gamma \left\langle nI_\gamma^2\right\rangle \int dt_i \left[AB(e_0,T,t_i)\right]_{\alpha\gamma}  \left[AB(e_0,T,t_i)\right]_{\beta\gamma} 
\label{eq:ABq}
\end{eqnarray}
which we would wish to average over the phase of the initial mean anomaly $M_0$ of the MBA.

Section~\ref{sec:impulse} describes the estimation of $\left\langle
  nI_\alpha^2\right\rangle$ for MBA regions.
Section~\ref{sec:propagation} derives the forms of $\matA(e_0,T,t_i),$
and $\matB(e_0,T).$  Section~\ref{sec:results} combines these and
summarizes the results and their implications.

\section{Impulse distribution}
\label{sec:impulse}
\subsection{Derivation}
Under the assumption that the distributions of masses and orbital
elements of MBAs are separable, the quantities needed to describe the
random walk of the asteroid's orbits can be expressed as an integral
over the absolute magnitude $H$ of the deflectors, the velocity $v$,
impact parameter $b,$ and polar/azimuthal angles $\theta,\phi$ of the
unit vector of the impulse direction $\bhat$:
\begin{eqnarray}
  \left\langle n I_\gamma^2 \right\rangle & = & \int
                                              dH\,dv\,db\,d\theta\,d\phi
                                             \frac{dN}{dH} \frac{dn}{dv\,db\,d\theta\,d\phi}
                                              I^2_\gamma(H,v,b,\theta\phi)
  \\
  & = & \int dH \frac{dN}{dH} \int dv
        \frac{dn}{dA\,dv\,d\theta\,d\phi} \hat b_\gamma^2(\theta,\phi)
        \int_{b_{\rm min}}^{b_{\rm max}} 2\pi b\,db\,
        \left(\frac{\sigma_M(H)}{bv}\right)^2 \\
  & = & 2\pi \int dH \frac{dN}{dH} \sigma^2_M(H)  \int dv
        \frac{dn}{dA\,dv} v^{-2} \left\langle\hat b_\gamma^2(v)\right\rangle
        \log(b_{\rm max}/b_{\rm min}).
\end{eqnarray}
The second row adopts the (numerically verified) assumption that the
distribution of impact parameter $b$ will always be $\propto dA=2\pi
b\,db$ in the range of $b$ of interest to us.  The mean rate of encounters
between a deflector-tracer pair, as a function of the impact velocity
and geometry, is $dn/dA\,dv\,d\theta\,d\phi,$ which we will determine
through numerical measurements with the orbits of all known MBAs. The
$H$ distribution of MBAs is $dN/dH.$ We denote as $\sigma_M(H)$
the RMS error in the mass of MBAs at $H$ from the value used in the ephemeris model.
We use $\sigma_M$ rather than $M$ since we are seeking the size of
deviations from the ephemeris model.  For MBAs that are not included
in the ephemeris model, $\sigma_M=M(H),$ the full mass of the deflector.

In the third line, we execute the integral over $b$ and introduce the
average geometry factors $\langle \hat b^2_\gamma\rangle$ of encounters as
a function of $v$.  It must be true that
\begin{equation}
  \left\langle \hat b_r^2 \right\rangle
  + \left\langle \hat b_\phi^2 \right\rangle
  + \left\langle \hat b_z^2 \right\rangle = 1,
\end{equation}
but equality of the three is not required.

For the values of $b_{\rm min}$ and $b_{\rm max},$ we adopt heuristics
for the nature of impulses of interest.  We set $b_{\rm max}$ by the
requiring that $b/v<1,$ such that the ``impulse'' is being applied
over a time shorter than $1/2\pi$ of the orbital period.  Events
lasting longer than this are either extended close encounters of 2
MBAs with very similar orbits---which are rare and can be identified
and modeled in advance; or they are slow, longer-range interactions
that would be adequately modeled by a multipole model of the
collective mass of the asteroid belts.  Thus we take $b_{\rm max}=v.$

For $b_{\rm min},$ we adopt the criterion that encounters at
sufficiently small $b$ will generate sufficiently large impulses $I$
that observations of the tracer will detect this individual impulse at
levels well above measurement noise.  This would mean that an
ephemeris model could include the mass of the deflector asteroid, and
this mass could be determined from fitting the data of the tracer.
Such encounters would thus no longer be considered source of
stochastic Brownian motion.  Finding all such encounters among known
MBAs is feasible, and in fact is forecasted for the next decade's LSST
observations by
\citet{negin}. The fitting of the tracer data for the mass of the
deflector is also a small computational effort.

If we crudely choose some threshold
$I_{\rm det}$ of impulse as being large enough to generate high-$S/N$
deflections on its tracer, then our condition becomes $M(H)/bv <
I_{\rm det},$ or $b_{\rm min}=M(H)/vI_{\rm det}.$  We then have
\begin{equation}
   \left\langle n I_\gamma^2 \right\rangle = 
2\pi \int dH \frac{dN}{dH} \sigma^2_M(H)  \int_{\sqrt{M(H)/I_{\rm
      det}}}^\infty \frac{dv}{v^2} 
        \frac{dn}{dA\,dv}  \left\langle\hat b_\gamma^2(v)\right\rangle
        \log \left[v^2 I_{\rm det}/M(H)\right].
        \label{eq:nI2}
      \end{equation}
The lower bound on the $v$ integral marks the speed below which
$b_{\rm max}<b_{\rm min}.$  Since $b_{\rm min}$ and $b_{\rm max}$
appear logarithmically, the estimated diffusion is relatively insensitive to choices of
their values.

\subsection{Numerical results}
\eqq{eq:nI2} reduces the calculation to a double integral over the
deflector $H$ and the tracer-deflector relative velocity $v.$  The
constituents of this calculation are estimated as follows.

\subsubsection{Encounter rates and geometries}
The pairwise interaction rate $dn/dA\,dv$ and the impulse direction
distributions $\langle \hat b^2_\gamma \rangle$ are estimated from a
numerical integration of all of the MBA orbits available
from the Minor Planet Center (MPC) as of 6 January 2025.   We retain as
potential deflectors the 1.26~million bodies with $1.8<a<4.2$~AU,
observations at multiple oppositions, and uncertainty values $U\le5.$
We designate each source as being an ``Inner'' MBA, with semi-major axes between the 4:1 and 3:1
mean-motion resonances of Jupiter; ``Middle'' MBA between the 3:1 and
5:2 resonances; ``Outer'' MBAs between the 5:2 and 2:1
resonances; the remaining ``Other'' objects are a small minority that
contribute very little to the total diffusion rate, and we will ignore.  Our working
assumption will be that the MPC objects are an unbiased sample of the
orbital-element distribution within each of the Inner, Middle, and
Outer belt regions.

From this full MBA list, we select a random subset of 5000 objects
from each of the Inner, Middle, and Outer belts to serve as a sample
of ``tracers.''  We record all of the passages of any tracer MBA
within 0.03~AU of any other ``deflector'' MBA (drawn from the full
catalog of 1.26~million) over a 10-year period.  The 5000 tracers per
belt region are enough to be representative of the statistics of
the region, but few enough that the calculation can be done easily
on a laptop computer.
We will estimate the constituents of  $\langle nI_\gamma^2\rangle$ in
\eqq{eq:nI2} separately for each of the nine combinations of tracers
in Inner, Middle, and Outer regions with deflectors from each region.

Using a simple leapfrog integrator with gravity from the Sun and the 8
major planets, we advance all MBA's orbits from their heliocentric
osculating elements at the epochs given by the MPC, to barycentric
state vectors on 1 May 2025.  Using $kd$-trees to accelerate
pair-finding, we locate all tracer-deflector pairs
that pass within 0.03~AU of each other during the $\pm1$-day period
around this initial epoch, assuming inertial relative motion during
this interval.  The circumstances of such encounters are saved: the
identities of the two MBAs involved, the time of closest approach, the
relative velocity $v,$ and the impact parameter vector $\vecb.$

The leapfrog integrator advances all 1.26 million state vectors by 2
days, and the process is repeated.  We continue searching for pairs at
2-day intervals until we have recorded 10 years' worth of encounters---17.7 million
total impulses.

\begin{figure}
  \centering
  \includegraphics[width=0.8\textwidth]{isq.pdf}
  \caption{Each panel shows statistics of the encounters between
    tracer MBAs from one region and deflector MBAs from another.  As a
    function of the relative velocity at impact, the three curves show
    the mean squared value of the radial, azimuthal, and vertical
    components of the impulse direction $\bhat.$  The horizontal
    dashed line at value $1/3$ would be the result for isotropic
    encounters, but we see that the azimuthal component of the impulse
    tends to be larger, with a variety of behaviors.  The gray regions
    trace the distribution $dN/dv$ of the rate of interactions per
    tracer per orbit per interval in $v.$  These curves have a common
    (arbitrary) normalization, so the relative heights properly
    reflect the frequency of encounters between different MBA
    regions.}
  \label{fig:rtz}
\end{figure}

For each pair of tracer-deflector regions (\eg\ Inner-Inner), we combine the list of
events with the counts of candidate tracers and deflectors to
calculate $dn/dA\,dv,$ the event rates per eligible pair; and the mean
geometry of the encounter, $\langle \hat b^2_\gamma \rangle,$ as a
function of $v$.  Figure~\ref{fig:rtz} plots these functions of $v$
for the Inner-Inner, Middle-Middle, and Outer-Outer scattering
events.  Here it is apparent that the $\hat b_\phi$ component is
larger than $\hat b_z$ and $\hat b_r,$ but there is significant
variation with MBA class and with $v$.  An important result is that
the quantity $v^{-2} (dN/dA\,dv)$ is always well-behaved as $v\rightarrow 0,$
so we do not need any special treatment to sum the integrals in
\eqq{eq:nI2}.  The $v$ integrand peak is in the range $0.03<v/v_c<0.1.$

\subsubsection{Asteroid properties}
\label{sec:mbaprops}
We crudely approximate the population as having a geometric albedo of 0.25,
density of 2500~kg~m$^{-3},$ and spherical shapes, which leads to the
conversion
\begin{equation}
  M(H) = 1.2\times10^{-17} M_\odot \times 10^{-0.6(H-15)}
\end{equation}

\begin{figure}
  \centering
  \includegraphics[width=0.8\textwidth]{mbacounts.pdf}
  \caption{Differential counts of MBAs cataloged by the MPC are
    plotted vs absolute magnitude $H$  
    for the Inner, Middle, Outer, and ``Other'' regions of
    the main belt, and for their total.  We assume that they are
    complete for $H\le17,$ and extrapolate to fainter sources using
    the functional form from \eqq{eq:lsstmba} shown as the dotted
    curve.}
  \label{fig:counts}
\end{figure}
We also require the $H$ distribution of potential deflectors.
Figure~\ref{fig:counts} plots our assumptions for $dN/dH.$  The MPC
catalog is likely close to complete in all three regions for $H\le17,$
so we will use the MPC counts directly in this regime.  Note that the
three regions have similar shapes of $dN/dH.$  For $H>17,$ we adopt
the functional form for the cumulative MBA counts given by
initial LSST predictions \citep{lsstbook}:
\begin{equation}
  N(<H) \propto \frac{10^{0.43(H-15.7)}}{10^{0.18(H-15.7)} +
    10^{-0.18(H-15.7)}},
\label{eq:lsstmba}
\end{equation}
normalizing this curve to the observed $H<17$ counts for each region.

Our integral for the uncertainty due to unmodelled MBA gravitation
requires the RMS error $\sigma_M(H)$ between the true mass $M$ and the
mass used (if any) in the ephemeris model.
The two largest asteroids, Ceres and Vesta, have masses determined to
high precision by the \textit{Dawn} spacecraft \citep{Ceres,Vesta}.  Thus despite holding
half of the $\approx 10^{-9} M_\odot$ mass of the asteroid belt, their $\sigma_M$
values\footnote{More precisely, the uncertainties in their $GM$ values
  relative to $GM_\odot$.} are 
$\le3\times10^{-15} M_\odot$ and
% Konilov papers: 1.2e-5 km^3/s^2 for Vesta, 4e-4 for Ceres
this unknown part of their contribution to other MBA's motion is 
unimportant.  Aside from a handful of other asteroids in
well-characterized binary systems or visited by spacecraft, most other knowledge of MBA masses
has and will come from mutual gravitational encounters.   An
individual asteroid's mass can be estimated from its measured effect  on the
few MBAs on which it imparts the largest impulses.  The accuracy
$\sigma_M$ of this determination will vary, depending upon the
geometry  and timing of
each deflector MBA's encounters, and the measurement errors on its tracers.  These
factors are independent of the mass of the deflector.  We will
assign a value $\sigma_M$ to the RMS uncertainty in MBA mass
attainable from following a few individual tracers, and calculate
the Brownian-motion noise as a function of $\sigma_M.$  We take
\begin{equation}
  \sigma_M(H) = \textrm{min}\left[ \sigma_M, M(H) \right],
\end{equation}
because a given MBA could have a mass determination accurate to
$\sigma_M$ if included in the ephemeris model, or if $M(H)<\sigma_M,$
it would be omitted from the ephemeris model.


\begin{figure}
  \plottwo{sigmaI.pdf}{sigmaM.pdf}
  \caption{Forecasts of uncertainties derived from LSST tracking of the known MBAs.  The left-hand plot shows the distribution of (log) uncertainties $\sigma_I$ on the amplitude of the impulse applied to the tracer MBA, incorporating forecasted LSST observations and marginalizing over the tracer's initial state vector.  The right-hand side shows the distribution of the uncertainties on the mass, $\sigma_M=(bv/2)\sigma_I,$ for deflector MBAs expected from LSST.  The histograms show the results from using only the single tracer yielding the lowest $\sigma_M$ for that deflector; combining the 10 best-$\sigma_M$ tracers; or combining all impulses at $b<0.03$~AU.  The RMS value of $\sigma_M$ over the deflector population is given in the legend.  Masses are in solar units, and velocities in units of the circular velocity of the deflector.}
  \label{fig:sigmaI}
\end{figure}

To give some context, a
comprehensive investigation of MBA masses from mutual scattering
events by \citet{goffin} reports typical $\sigma_M\approx
5\times10^{-13}$ for 30 MBAs. But we can expect substantial improvement
in the LSST era when a millions of potential tracer asteroids attain
mas-level astrometry. \citet{negin} use a simulated LSST sequence of observations to estimate the impulse (and mass) uncertainties from mutual encounters of the known asteroids that will occur during years 2--9 of its 10-year survey.  The tracer MBAs for those events are essentially an ubiased selection from the known MBAs, and the left-hand plot of Figure~\ref{fig:sigmaM} shows the distribution of the uncertainty $\sigma_I$ derived from each tracer's LSST data after marginalizing over its initial state vector.  This histogram marginalizes over the apparent magnitude and observing conditions of the tracer, the date of the encounter, and the direction $\bhat$ of the impulse.  The right-hand plot of Figure~\ref{fig:sigmaI} forecasts the distribution of $\sigma_M$ that we can expect for deflectors.  This forecast is made as follows: for each of the 5000 deflectors that we have selected from each MBA region, we have a list of all of its encounters with known MBAs with $b<0.03$~AU.  For each such encounter, we select at random one $\sigma_I$ from the distribution found by \citet{negin}, and calculate $\sigma_M=(bv/2)\sigma_I.$  For each deflector, we then plot the histogram of (a) the individual encounter with lowest $\sigma_M$; (b) the combined constraint $\left(\sum \sigma_{M,i}^2\right)^{-1/2}$ from the 10 lowest $\sigma_M$ encounters for that deflector; and (c) the combined constraint for all $b<0.03$~AU encounters (typically $\sim1000$).  The RMS $\sigma_M$ using LSST data from the 10 most-informative tracer encounters with a given deflector is estimated as $\sigma_M=2.6\times10^{-14}M_\odot.$ There are $\approx450$ MBAs with mass above this forecasted RMS $\sigma_M.$  This is a conservative estimate in that we ignore any observational information before LSST, and we also do not consider information from the $\approx10\times$ more MBAs that LSST will discover.  



\subsubsection{Individually detectable impulses}
Examining the $\sigma_I$ distribution forecasted for LSST data on tracers (Figure~\ref{fig:sigmaI}, we find that for impulses occurring near the midpoint of the survey, $\approx80\%$ of all cases yield $\sigma_I<10^{-4}\,{\rm m}\,{\rm
  s}^{-1},$ some an order of magnitude lower.  We conservatively
assume that any encounter producing  an impulse $>5\times$ this value, $I_{\rm det}=5\times 10^{-4}\,{\rm m}\,{\rm
  s}^{-1},$ can be used to directly constrain the relevant deflector and include it in the ephemeris model.
When normalized by $v_c$ yields $I_{\rm  det}=10^{-8}$ at 10\% accuracy across the main belt.

\section{Propagation to position shifts}
\label{sec:propagation}

\subsection{Element shifts from impulses}
\label{sec:elements}

Five of our orbital element perturbations $\vecq$ will be taken from constants of the motion.  The mean anomaly $M$ is not appropriate as the sixth, time-dependent element of $\vecq$ because the derivative $dM/dI$ can diverge as $e_0\rightarrow 0.$ Instead we introduce $\tau = M+\lop,$ which we will show does have a shift $\Delta\tau$ during an impulse that has finite and linear response to $\vecI$.  Between impulses, $\tau$ advances with the mean motion as $a^{-3/2}t.$  We define a coordinate system that is uniformly rotating with the unperturbed $\tau_0=t+M_0.$ The two smoothly rotating unit vectors $\uhat_\parallel=(\cos \tau_0, \sin \tau_0)$ and
$\uhat_\perp=(-\sin \tau_0, \cos \tau_0)$ satisfy $\uhat_\parallel
\times \uhat_\perp = \zhat,$ and we will use subscripts $\parallel$
and $\perp$ to represent projections onto these components.  In
particular, the components $(e_\parallel,e_\perp)$ of the ellipticity
vector $\vece$ have the values $e_\parallel=e\cos M$ and $e_\perp=-e\sin M$ in the unperturbed orbit.
The true anomaly $\nu,$ radius $r$, and azimuthal angle $\phi,$ and
the velocity components of the initial orbit are, to first order in $e$,
\begin{eqnarray}
  M & = & M_0 + t = \tau_0,  \nonumber \\
  \nu & = & M - 2 e_\perp \nonumber \\
  \phi & = & \nu + \lop = \tau - 2e_\perp, \nonumber\\
  r & = & 1-e_\parallel, \nonumber \\
  v_r & = & -e_\perp \nonumber \\
  v_t & = & 1 + e_\parallel
            \label{eq:kepler}
\end{eqnarray}

The first element adopted for $\vecq$ is the shift $\Delta a$, derivable from the orbital energy $E=-1/2a.$
\begin{eqnarray}
  \Delta E & = &  \vecv \cdot \vecI + O(I^2) \\
  \quad \Rightarrow \quad \Delta a & = & 2\left(v_r I_r + v_\phi I_\phi\right) = -2e_\perp I_r + 2(1+e_\parallel) I_\phi.
  \label{eq:da}
\end{eqnarray}
The initial angular momentum $\mathbf{L} = \sqrt{1-e^2}\,\zhat \approx \zhat$ is altered by
\begin{eqnarray}
  \Delta \mathbf{L} & = & \vecr \times \vecI = (1-e_\parallel) I_\phi \zhat - (1-e_\parallel) I_z \phat \\
  \quad \Rightarrow \quad \Delta \Lhat_\parallel & = & -2e_\perp I_z
  \label{eq:dLpar}\\
  \Delta \Lhat_\perp & = & -(1-e_\parallel)  I_z
                  \label{eq:dLperp}
\end{eqnarray}
We adopt the changes in direction $(\Delta\Lhat_x, \Delta\Lhat_y)$ as the next two elements of $\vecq$ specifying the inclination and ascending node $\Omega$ of the perturbed orbit.  These are a rotation of $(\Delta\Lhat_\parallel, \Delta\Lhat_\perp)$ by the angle $\tau.$  The conversion from the first line above to the subsequent two makes use of the decomposition $\phat=\uhat_\perp + 2e_\perp \uhat_\parallel$ derivable from the value of $\phi$ in Equations~(\ref{eq:kepler}).

The eccentricity vector $\vece = \vecv \times (\vecr \times \vecv) - \rhat$ is altered by the impulse according to
\begin{eqnarray}
  \Delta\vece & = & (2\rhat - rv_r \phat) I_\phi - \phat I_r \\
\label{eq:depar}
  \quad \Rightarrow \quad \Delta e_\parallel & = &  -2e_\perp I_r + 2 I_\phi \\
\Delta e_\perp & = & -I_r - 3e_\perp I_\phi.
\label{eq:deperp}
\end{eqnarray}
We adopt $\Delta e_x$ and $\Delta e_y$ as two additional components of $\vecq,$ again related through a rotation by $\tau$ to the quantities in Equations~(\ref{eq:depar}) and (\ref{eq:deperp}).

The final element of $\vecq$ will be the shift $\Delta\tau$ induced in
$M+\lop$ by the impulse.  This can be obtained by enforcing the
conditions that the radius $r$ or the azimuthal angle $\phi$ must
remain constant during the impulse.  To find the latter, we need a formula for $\nu(M)$ as a power series in $e.$  Standard formulae in the literature lead to:
\begin{eqnarray}
  \nu & = & M + 2e \sin M + \frac{5e^2}{4} \sin 2M + O(e^3) \\
  \label{eq:nue2}
    & = & M + 2(e\sin M) + \frac{5}{2} (e \sin M) (e \cos M) + O(e^3) 
\end{eqnarray}
We wish to find the first-order shift in $\phi=\nu + \lop$ and set it
to zero.  But the derivatives of $M$ and $\lop$ with impulse can become infinite.  Instead we introduce a perturbation $\Delta\tau = \Delta(M+\lop).$ We can now express several post-impact quantities as deviations from the unperturbed quantities (subscripted with 0):
\begin{eqnarray}
  M & = & \tau-\lop = \tau_0 -\lop + \Delta\tau \\
  e \sin M & = & e \sin (\tau_0 -\lop) + \Delta\tau \left[ e \cos  (\tau_0 -\lop)\right] \nonumber \\
\label{eq:esinM}
    & = & -e_{0\perp}-\Delta e_\perp + e_{0\parallel} \Delta\tau +O(e^2) \\
  e \cos M & = & e \cos (\tau_0 -\lop) - \Delta\tau \left[ e \sin  (\tau_0 -\lop)\right] \nonumber\\
\label{eq:ecosM}
    & = & e_{0\parallel}+\Delta e_\parallel + e_{0\perp} \Delta\tau +O(e^2) \\
\Rightarrow \quad \phi =\nu+\lop & = & \tau_0 + \Delta \tau + 2( -e_{0\perp}-\Delta e_\perp + e_{0\parallel} \Delta\tau )
                              + \frac{5}{2}\left( -e_{0\perp}\Delta e_\parallel-e_{0\parallel}\Delta e_\perp\right) + O(e^2).
\label{eq:phi}
\end{eqnarray}
Forcing $\phi$ to be unchanged during the impulse requires
\begin{eqnarray}
  \Delta\tau & = & \frac{5 e_\perp}{2} \Delta e_\parallel + \left(2-\frac{3e_\parallel}{2}\right) \Delta e_\perp \\
             & = & \left(-2+\frac{3e_\parallel}{2}\right) I_r - e_\perp I_\phi.
                   \label{eq:dtau0}
\end{eqnarray}
We have dropped the 0 subscripts on $e_\parallel,e_\perp$ at this point for brevity---they will refer to the values for the unperturbed orbit at the time of impulse, unless noted otherwise.

After the impulse, the mean anomaly $M$ advances at a rate of $a^{-3/2}t,$ meaning that an additional term $\Delta\tau = -3\Delta a (t-t_i)/2$ accrues by time $t$. The total perturbation of $\tau$ from the nominal value $t+M_0$ becomes
\begin{equation}
  \Delta\tau= \left[-2+\frac{3e_\parallel}{2} +3 e_\perp (t-t_i)\right] I_r  - \left[3(1+e_\parallel)(t-t_i) + e_\perp\right] I_\phi.
  \label{eq:dtau}
\end{equation}

We now have all the information we need to create the $\matA$ matrix.
Rotating the $\vecq$ elements back into the Cartesian frame, we
obtain, to $O(e)$,
\begin{equation}
  \left( \begin{array}{c}
           \Delta a \\ \Delta\tau \\ \Delta e_x \\ \Delta e_y \\ \Delta \Lhat_x \\ \Delta\Lhat_y
         \end{array}\right) = \left( \begin{array}{ccc}
 2e\sin\tau & 2 + 2e\cos\tau & 0 \\
 -2+e\left[1.5\cos\tau-3(T-t_i)\sin\tau\right] & e\sin\tau -3(1+e\cos\tau)(T-t_i) & 0\\
 -\cos\tau-2e\sin^2\tau & 2\cos\tau - 1.5e\sin 2\tau & 0\\
  -\sin\tau + e\sin 2\tau & 2\sin\tau-3e\cos^2\tau & 0 \\
0 & 0 & \sin\tau + 0.5 e \sin 2\tau \\
 0 & 0 &  -\cos\tau +e(3-\cos 2\tau)/2 \\
 \end{array}\right)
  \left(\begin{array}{c} I_r \\ I_\phi \\ I_z \end{array}\right)
\end{equation}
We can now do the time integral and sum over impulse directions in \eqq{eq:Cqjk}, and average over $\tau$ to average over the orbital phase of the impulses. The diagonal elements are
\begin{eqnarray}
  \Var(a) & = & 4T \nIphi \nonumber\\
  \Var(\tau) & = & 4T\nIr + 3T^3\nIphi \nonumber\\
  \Var(e_x) =  \Var(e_y) & = & \nIr/2+2\nIphi \nonumber\\
  \Var(\Lhat_x) =  \Var(\Lhat_y) & = & \nIz/2
\label{eq:Cq}
\end{eqnarray}
All of the off-diagonal terms (covariances) are either zero or $O(e)\left\langle nI^2\right\rangle,$ and they result in observable consequences that are smaller by $\left\langle e^2 \right\rangle$ than the diagonal terms' contributions.

% \begin{equation}
%   \matC^2 = \left(\begin{array}{cccccc}
%     4T \nIphi & -3T^2\nIphi & e\left(\nIr + 5\nIphi\right) & e\nIr & 0 & 0 \\
% -3T^2\nIphi & 4\nIr + 3T^3\nIphi & -e(11T\nIr/4 + 15T^2\nIphi/4) & e\left[(2T-3T^2/4)\nIr+(-T+9T^2/4)\nIphi\right] & 0 & 0\\
%  e\left(\nIr + 5\nIphi\right) & -e(11T\nIr/4 + 15T^2\nIphi/4) & \nIr/2+2\nIphi & 0 & 0 & 0 \\
%  e\nIr &  e\left[(2T-3T^2/4)\nIr+(-T+9T^2/4)\nIphi\right]  & 0 & \nIr/2+2\nIphi & 0 & 0 \\
% 0 & 0 & 0 & 0 & \nIz/2 & 0 \\
% 0 & 0 & 0 & 0 & 0 \nIz/2 \\
%\end{array}\right)
% \end{equation}
 \subsection{Astrometric and ranging shifts from elements shifts}
\label{sec:observe}
The radial coordinate of a Keplerian orbit is $r=a(1-e\cos E),$ and an expansion of the eccentric anomaly $E$ in powers of $e$ is
\begin{equation}
  \label{eq:Ee2}
  \cos E  =  e\cos M - (e \sin M)^2 + O(e^3).
\end{equation}
From this we can derive the perturbations to $r$ as a function of our chosen basis for perturbations to the orbital elements.  Making use of Equations~(\ref{eq:ecosM}) and (\ref{eq:esinM}) we obtain
\begin{equation}
  \Delta r = (1-e\cos\tau) \Delta a + e\sin\tau \Delta\tau +  (-\cos\tau -2e\sin^2\tau) \Delta e_x + (-\sin\tau+e\sin 2\tau) \Delta e_y.
\end{equation}
If we square this expression, average over the orbital phase $\tau$ of the observation, and retain leading order in $e$, we get the variance of the radial coordinate from the unperturbed orbit.  All of the terms involving covariances between orbital elements either average to zero over an orbit, or are suppressed by $e^2$ relative to other terms.  We are left with
\begin{eqnarray}
  \Var(r) & \approx & \Var(a) + \frac{1}{2}\Var(e_x) + \frac{1}{2}\Var(e_y) + \frac{\langle e^2\rangle}{2} \Var(\tau) \\
          & \approx & \frac{T}{2} \nIr + \left( 6T + \frac{3\langle e^2 \rangle}{2} T^3\right) \nIphi.
                      \label{eq:varr}
\end{eqnarray}
The second line substitutes the results from Equations~(\ref{eq:Cq}).  We retain the final term because, as $T>>1,$ this precession term that scales with $\langle e^2 \rangle T^3$ becomes comparable to or larger than the other terms that scale as $T$.  The $T$-scaling terms all arise from epicyclic motions that are periodic with the orbit.

The heliocentric azimuthal angle $\phi$ is given by \eqq{eq:phi}.  We can similarly square the perturbation and average over orbital phase and population factors to get
\begin{eqnarray}
  \Var(\phi) & \approx & \Var(\tau) + 2\Var(e_x) + 2\Var(e_y) \\
             & \approx & 6T\nIr + \left(8T+3T^3\right) \nIphi.
                         \label{eq:varphi}
\end{eqnarray}
The heliocentric latitude (or equivalently, polar angle) has a deviation
\begin{eqnarray}
  \Delta\theta & = & -\Delta\Lhat_x \frac{x}{r} - \Delta\Lhat_y \frac{y}{r} \\
  \Rightarrow \quad \Var(\theta) & \approx & \langle\cos^2\tau\rangle \Var(\Lhat_x) + \langle\sin^2\tau\rangle \Var(\Lhat_y) \\
               & \approx & \frac{T}{2} \nIz.
                           \label{eq:vartheta}
\end{eqnarray}

Equations~(\ref{eq:varr}), (\ref{eq:varphi}), and (\ref{eq:vartheta}) comprise our desired result.  Several characteristics are noteworthy:
\begin{itemize}
\item $r$ is in units of $a_0$ and $\theta, \phi$ are angles.  To convert their variances into linear displacements in the $r,\phi,z$ directions, multiply them all by $a^2,$ or equivalently scale the standard deviations by $a.$
\item The terms proportional to $T$ show typical diffusive growth $\Delta \propto \sqrt{T},$ and arise from perturbations to the 5 constants of Keplerian motion.  They represent epicyclic alterations to the original orbit that grow in amplitude as $\sqrt{T}.$  The $T^3$ terms dominate $\Var(\phi)$ by the end of a full orbit, resulting from accumulating delays/advances $\Delta\tau$ in orbital phase $M$ as $a$ and the mean motion diffuse.  Thus it is the energy-changing impulses, namely $I_\phi,$ that cause the most Brownian motion.
\item The radial variance also becomes dominated by the $T^3$ term if $e>0,$ as this term represents advances/delays in the radial epicycle phase.  This is the only part of the Brownian motion that whose leading order in $e$ is non-zero, \ie\ which we would have missed by assuming circular initial orbits.
\item The RMS vertical displacement grows strictly as $T^{1/2}.$
\item It will be generally true that $\Var(\phi) > \Var(r) > \Var(\theta),$ a trend exacerbated by the empirical observation (Figure~\ref{fig:rtz}) that impulses favor the $\phat$ direction.
\item There are no significant covariances among the $r,\theta,$ and $\phi$ displacements at the population-averaged level.
  \item For Earth-based observers, the deviations in $(r,\theta,\phi)$ from the nominal orbit will be some calculable, time-dependent, nearly-orthogonal transformation of the heliocentric deviations derived above, so the results will not be grossly different aside from bringing the three directional components closer together in amplitude.
\end{itemize}

\section{Results and conclusions}
\label{sec:results}
We can now propagate the numerical estimates for $\left\langle nI_\gamma^2\right\rangle$ from Section~\ref{sec:impulse} through the analytical observational consequences in Section~\ref{sec:observe} to yield estimates of the unmodelled RMS shifts of MBAs from each region.  We assume the following parameters:
\begin{itemize}
\item The deviations are accumulated over a time $T=10$~yr.
\item The population has $\langle e^2 \rangle=0.03,$ which is the value for the cataloged MBAs of the Inner and Middle belts, and a slight overestimate for the Outer belt.
\item The $5\sigma$ impulse size $I_{\rm det}$, at which we deem it possible to isolate the effect of a single deflector on a tracer and hence usefully include that deflector's mass in the ephemeris model, is taken to be $I_{\rm det} = 10^{-8}v_c.$ In practice this will depend on the geometry of the impulse and on the quality of the tracer's observations.  But we find that changing $I_{\rm det}$ by a factor of 10 alters the noise from unmodeled impulses by $<20\%.$
  \item We will vary $\sigma_M,$ the RMS uncertainty on the mass of modelled asteroids, such that the number $N$ of MBAs  with $M>\sigma_M$ varies from 100 to 10,000.  The number of MBAs included in the ephemeris model would be $\approx N.$
\end{itemize}


\begin{figure}
  \centering
  \includegraphics[width=0.6\textwidth]{result.pdf}
  \caption{Each line shows the forecast of the variance of the
    azimuthal position of a Middle Main Belt asteroid caused by discrete
    encounters with absolute magnitude $H$ greater than the value on
    the (lower) $x$ axis, \ie\ the cumulative noise from all MBAs
    smaller than the plotted $H_{\rm min.}$  The upper axis gives mass
    instead of $H$.  The left axis gives values in meters, the right
    the apparent shift in milliarseconds for a heliocentric observer.
    Each line is for a different assumption about which asteroids' deflections are
    part of the ephemeris model instead of being considered as noise: the $\sigma_M$
    value gives the RMS mass uncertainty on modelled MBAs, and the $N$
    is the number of MBAs with $M>\sigma_M$ that could be usefully
    modelled.}
  \label{fig:result}
\end{figure}

Figure~\ref{fig:result} plots the resulting noise level $\sigma_\phi=\sqrt{\Var{\phi}}$ of the calculation for the Middle regions of MBAs, showing both the linear RMS displacements on the left axis and the angular displacements on the right axis.  Each point on the curve is the $\sigma_\phi$ attributable to deflector MBAs that are smaller (fainter) than the $H$ value on the lower axis, or the mass on the top axis.  Thus the left-most point on each curve marks the total noise integrated over the full population of deflectors. Different color lines show the results for the values of $N$ and $\sigma_M$ as marked astride each line.  Notable features of these result are:
\begin{itemize}
\item We find that \emph{the linear displacements are nearly identical for Inner, Middle, and Outer regions of the asteroid belt.}  The angular displacements vary slightly, inversely to their distance from the Sun.
\item We plot the azimuthal noise.  \emph{The radial noise is $\approx 7\times$ lower than the azimuthal noise, and the vertical Brownian motion is $\approx45\times$ lower than the azimuthal, on average.}
\item For longer time periods $T$, the azimuthal noise will scale as
$T^{3/2},$ the radial noise slightly below this rate, and the vertical
noise as $T^{1/2}.$
\item The largest contributions to the unmodelled Brownian motion are
made by deflectors with mass and $H$ at the ``shoulder'' of each
curve, which will also correspond to the MBAs with masses $M\sim
\sigma_M$.  Above this mass, the noise per deflector is constant but
the deflectors become scarce at larger $M.$ Below this mass, $dN/dH$
is rising to fainter/smaller bodies, but the factor $M^2(H) dN/dH$
that determines the Brownian motion is dropping.
\end{itemize}

We take as representative of the present state of the art the case
where the $N=100$ most massive MBAs (aside from Ceres and Vesta) have
masses determined to $\sigma_M\approx 10^{-12.7} M_\odot$ by mutual
encounters.  Our calculations suggest then that a 10-year ephemeris
including these 100 MBAs will accrue RMS errors slightly under 1~km in
the azimuthal position, or 0.4~mas in observed ecliptic longitude.
RMS range and vertical ephemeris errors would be closer to 100 and 20
meters, respectively.  We find that, roughly speaking, $\sigma_\phi$
drops with the number of modeled MBAs as $1/N,$ and
\begin{equation} \sigma_\phi \equiv \sqrt{\Var{\phi}} \approx
(100\,\textrm{m}) \times \left(\sigma_M /
10^{-14}M_\odot\right)^{0.6}.
  \label{eq:varscale}
\end{equation}
 In the next decade, LSST will significantly
increase the number $N$ and lower $\sigma_M$ of deflector MBAs with
usefully measured masses from mutual encounters by tracking hugely
more potential tracers.  The conservative estimate given in Section~\ref{sec:mbaprops} is that LSST will lower $\sigma_M$ to $\approx10^{-13.7},$ which our calculations indicate lowers the $\sigma_\phi$ values to $\approx250$~m or 0.1~mas.

We need to check whether our supposition that impulses $I>I_{\rm det}$
would stand out above the measurement noise of a single tracer remains
true once we include the noise induced by the other unmodelled
impulses.  Focusing on the effects of a tangential impulse of
$I_\phi=I_{\rm det}/\sqrt{2},$ \eqq{eq:dtau} predicts an orbital phase
shift of $\Delta\tau \approx 3TI_\phi$ at time $T$ past the impulse,
translating directly into a position shift $\Delta\phi$ of the same
size as per \eqq{eq:phi}.  Consider a survey of duration $T,$ and a
single impulse $I_{\rm det}$ applied at time $T/2.$ If
$I_\phi=10^{-8}/\sqrt{2}$ and $T=10$~yr, then
$\Delta\phi\approx35$~mas.  This is nearly $100\times$ larger than the
stochastic $\sigma_\phi$ expected from unmodelled masses at the
current state of knowledge.  For primarily radial or vertical
impulses, the astrometric signal will be smaller, but the Brownian
motion noise is lower too.  Since $\Delta\phi/\sigma_\phi \propto
T^{-1/2},$ this signal-to-noise ratio would still be $\gg 5$ for
surveys of several decades.  We conclude that our choice of $I_{\rm
det}$ is not ruined by the presence of Brownian motion noise on the
tracer asteroids, rather it will be driven by the observational
uncertainties.

\subsection{Implications for future investigations}
  
An important question is: will the positional noise due to unmodelled MBA encounters degrade inferences being made with measurements of tracer MBAs?  If the Brownian noise is below other sources of measurement noise, then the answer must be ``no.''
The per-epoch uncertainties on solar-system object astrometry in Gaia DR3\footnote{We take one Gaia transit as an ``epoch'' here.} are reported to be as low as 0.3~mas for very bright sources ($G=13$), rising to $\approx10$~mas at its $G=21$ limit where most detections lie \citep[][Figure 6]{gaiass3}.  The overall RMS residual to the orbit fits are 5~mas (this is in the ``along-track'' direction).  Thus the Brownian motion of $\approx0.4$~mas we estimate for $\phi$ (over a 10-year period) would be important for the single-epoch error budget of only the brightest Gaia MBAs.  
The Brownian motion is, however, highly correlated across epochs, whereas measurement errors are independent and decrease as the square root of the number of epochs, which is $\approx20$ over 3 years for a typical MBA in the Gaia DR3 catalog.  The Brownian motion might become relevant for analysis of a significant fraction of Gaia MBAs during analysis of the complete 11-year data.  But with LSST estimates of deflector masses in hand, with $\sigma_\phi\approx0.1$~mas, the (retrospective) use of Gaia data will probably have negligible uncertainty from Brownian motion.

For LSST, the measurement errors on MBA positions will have a floor of 1--2~mas per epoch set by refraction from atmospheric turbulence \citep{willow,trojans}.  Objects with apparent magnitude $m\gtrsim20$ will have larger per-epoch errors due to photon noise in the images.  There will, however, be hundreds of observations taken of each MBA over 10~yr, meaning that Brownian motion might be a significant contributor to inferences from the dynamics of brighter tracer MBAs, if our $\sigma_M$ levels were fixed near current knowledge.

On the other hand, LSST will greatly improved the astrometric precision for $\sim 10^7$ potential tracer MBAs, significantly lowering $\sigma_M,$ so a self-consistent analysis should consider the per-epoch contribution of unmodelled Brownian motion to grow to $\approx0.1$~mas by the end of the survey.  Considering the coherence of the Brownian noise over $O(100)$ LSST observations, the Brownian noise could become comparable to measurement noise for inferences using the brighter tracer MBAs in LSST data, but will remain unimportant for the great majority of LSST's measured MBAs.

An even more ambitious proposal by \citet{occult} is to track $\approx10^6$ MBAs down to diameters of $\approx1$~km by watching them occult Gaia stars.  The timing of each occultation can locate the center of mass of the target MBA along its direction of motion, with most of the information available from objects yielding uncertainties of 50--100~m on this position.  The Brownian motion noise over a 10-year baseline will in this case become comparable to the per-epoch measurement errors.  Reducing the typical mass uncertainty of individual asteroids to $\sigma_M\approx 10^{-15.4} M_\odot$ would be needed ($N=3000$) to reduce the Brownian noise to negligible levels for this survey.

The extent to which the Brownian motion can confound an inference also depends on how similar the spatio-temporal dependence of the Brownian motion is to the signal under test.  If, for example, one is searching for the tidal force of a Planet X, this will be manifested primarily as a collective apsidal precession and a quadrupolar deviation of the orbit from its ellipse that does not evolve with time.  These are quite distinct from the random walk in $\Delta\tau$ that is the dominant effect of Brownian motion.  The Brownian noise will be somewhat orthogonal to the signal under study in the data space, and therefore cause less degradation to the Planet X inference than the per-epoch $\sigma_\phi$ would suggest.  Another example would be a gravitational perturbation whose signature is primarily a nodal precession---we have seen that Brownian motion out of the orbital plane is $>40\times$ smaller than the $\phi$ component.   At the other extreme, if one were trying to measure the Yarkovsky effect (which, to our knowledge, has never been detected on an MBA), this will be primarily realized as a time-invariant tangential force (torque) whose amplitude and direction depend on the (unknown) spin axis, the spin rate, and the radiant and thermal properties of the surface. This will lead to a linear increase or decrease in $\Delta\tau$ over the observation period.  This signal may be  difficult to discern from the Brownian motion's random walk in $\Delta\tau$ over the same interval.  The ratio of the Yarkovsky signal to the Brownian motion will increase as $\sqrt{T}$ and will also scale with the inverse diameter of the source.  There will be some nominal diameter below which the unknown Yarkovsky signal leads to larger observable $\sigma_\phi$ than the $\sigma_\phi$ we have derived from Brownian motion.  This will be good news for those wishing to measure individual bodies' Yarkovsky effect, and bad news for those wishing to use these bodies as test particles for other gravitational or non-gravitational forces.

In conclusion, the Brownian motion from unmodelled or mis-modelled mutual impulses between MBAs causes RMS azimuthal displacements at the 1~km scale over 10 years, with current knowledge of the masses of large asteroids.  This is large enough to become a significant part of the error budget for many inferences from the larger/brighter asteroids being used as tracer particles from Gaia or LSST images.  If, however, we can use mutual events from LSST to constrain the masses of the $\approx500$ largest  asteroids with $S/N\gtrsim1,$ then the unmodelled Brownian motion shrinks to $\approx250$~m (or 0.1~mas) for MBAs, which becomes insignificantly small for most uses of Gaia or LSST astrometry.  Occultation-based astrometry might, at this stage, have measurement errors that are similar to the uncertainties from Brownian motion on decade time scales.


\begin{acknowledgments}
  This work was supported by NSF grant AST-????.
  This research has made use of data and/or services provided by the International Astronomical Union's Minor Planet Center. 
\end{acknowledgments}

\newpage
\bibliographystyle{aasjournal}
\bibliography{references}


\end{document}
